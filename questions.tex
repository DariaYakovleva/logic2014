\documentclass[12pt,a4paper,oneside]{book}
\usepackage{fullpage}
\pagenumbering{gobble}
\usepackage[utf8]{inputenc}
\usepackage[english,russian]{babel}
\begin{document}

\begin{center}
\begin{Large}Программа курса <<Математическая логика>>\end{Large}\\
ИТМО, группы 2536-2539, осень 2014 г.
\end{center}

\begin{enumerate}
\item Исчисление высказываний. Общезначимость, доказуемость и выводимость. Теорема о дедукции для исчисления высказываний.
\item Теорема о полноте исчисления высказываний.
\item Интуиционистское исчисление высказываний. Теорема Гливенко (группы 2538, 2539 --- с доказательством). 
Решетки, булевы и псевдобулевы алгебры. Топологическая интерпретация интуиционистской логики.
\item Модели Крипке. Полнота интуиционистского исчисления высказываний в моделях Крипке и 
псевдобулевых алгебрах. Дизъюнктивность интуиционистского исчисления высказываний. Нетабличность интуиционистской логики.
\item Исчисление предикатов. Общезначимость и выводимость. Теорема о дедукции в исчислении предикатов.
\item Теорема о полноте исчисления предикатов.
\item Теории первого порядка, структуры и модели. Аксиоматика Пеано. Формальная арифметика. 
\item Рекурсивные функции и отношения. Функция Аккермана. Существование рекурсивных функций,
не являющихся примитивно-рекурсивными (группы 2538, 2539 --- с доказательством). 
\item Представимость функций в формальной арифметике. Бета-функция Гёделя. 
Представимость рекурсивных функций в формальной арифметике.
\item Выразимость отношений. Гёделева нумерация. Выводимость и рекурсивные функции.
\item Непротиворечивость и $\omega$-непротиворечивость. Первая теорема Гёделя о неполноте арифметики.
\item Вторая теорема Гёделя о неполноте арифметики, $Consis$, условия выводимости Гильберта-Бернайса.
\item Теория множеств. Аксиоматика Цермело-Френкеля.
\item Ординальные числа. Операции над ординальными числами. 
\item Кардинальные числа. Теорема Лёвенгейма-Сколема. Парадокс Сколема.
\item Теорема о непротиворечивости формальной арифметики.
\end{enumerate}

\end{document}