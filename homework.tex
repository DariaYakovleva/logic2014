\documentclass[11pt,a4paper,oneside]{book}
\usepackage[utf8]{inputenc}
\usepackage[english,russian]{babel}
\usepackage{amssymb}
%\usepackage{mathabx}
\usepackage[left=2cm,right=2cm,top=2cm,bottom=2cm,bindingoffset=0cm]{geometry}
\usepackage{bnf}
\newcommand{\lit}[1]{\mbox{`\texttt{#1}'}}
\newcommand{\ntm}[1]{<\mbox{#1}>}
\begin{document}

\begin{center}
\begin{Large}Домашние задания по курсу <<Математическая логика>>\end{Large}\\
ИТМО, группы 2536-2539, осень 2014 г.
\end{center}

%Задания с дробными номерами необязательны, их решение, впрочем, будет давать бонус
%для зачета. 
Для всех программ кодировка выходных файлов должна быть либо CP1251,
либо UTF8.

\begin{enumerate}
\item[1] Написать программу, проверяющую доказательства в исчислении высказываний на 
корректность. Входной файл представляет из себя последовательность высказываний, по 
высказыванию на строку. Высказывания удовлетворяют приведенной ниже грамматике. 
\begin{bnf}\begin{eqnarray*}
\ntm{выражение} &::=& \ntm{дизъюнкция} | \ntm{дизъюнкция} \lit{->} \ntm{выражение}\\
\ntm{дизъюнкция} &::=& \ntm{конъюнкция} | \ntm{дизъюнкция} \lit{|} \ntm{конъюнкция}\\
\ntm{конъюнкция} &::=& \ntm{отрицание} | \ntm{конъюнкция} \lit{\&} \ntm{отрицание}\\
\ntm{отрицание} &::=& (\lit{A} \dots \lit{Z}) \{\lit{0}\dots\lit{9}\}^* | \lit{!} \ntm{отрицание} | \lit{(} \ntm{выражение} \lit{)}
\end{eqnarray*}\end{bnf}%

Пробелы в строке должны игнорироваться.
Результатом работы программы должен быть проаннотированный текст доказательства,
каждая строка должна соответствовать грамматике:
\begin{bnf}\begin{eqnarray*}
\ntm{строка} &::=& \lit{(} \ntm{номер} \lit{) } \ntm{выражение} \lit{ (} \ntm{аннотация} \lit{)}\\
\ntm{аннотация} &::=& \lit{Сх. акс. } \ntm{номер} \\
                &|& \lit{M.P. } \ntm{номер}\lit{, }\ntm{номер}\\
                &|& \lit{Не доказано}\\
\ntm{номер} &::=& (\lit{0}\dots\lit{9})^+
\end{eqnarray*}\end{bnf}%

Выражение не должно содержать пробелов, номер от выражения и выражение от аннотации должны
отделяться одним пробелом. Выражения в доказательстве должны нумероваться подряд
натуральными числами с 1. Если выражение $\gamma_n$ получено из 
$\gamma_i$ и $\gamma_j$, где $\gamma_j \equiv \gamma_i\rightarrow\gamma_n$
путем применения правила Modus Ponens, то аннотация должна выглядеть как 
\lit{M.P. $i$, $j$}, обратный порядок номеров не допускается.

Уделите внимание производительности: ваша программа должна проверять доказательство в 
5000 выражений (общим объемом $1$Мб) на Intel Core i5-2520M ($2.5$ GHz) за несколько секунд.

\item[2] Написать программу, преобразующую вывод $\Gamma, \alpha \vdash \beta$ в вывод
$\Gamma \vdash \alpha \rightarrow \beta$.
Первой строкой входного файла должна являться строка, перечисляющая гипотезы, использованные 
в выводе, и выводимое утверждение. На следующих строчках входного файла перечислены 
высказывания
исходного вывода. Высказывания удовлетворяют грамматике из предыдущего задания,
первая строка соответствует следующей грамматике:
\begin{bnf}\begin{eqnarray*}
\ntm{заголовок} &::=& \{\ntm{выражение} \lit{,}\}^* \ntm{выражение} \lit{|-} \ntm{выражение}
\end{eqnarray*}\end{bnf}%

Символ `\texttt{|}' имеет ASCII-код $124_{10}$.

Результатом работы программы должен быть текст, содержащий преобразованный вывод.
Формат выходного файла совпадает с форматом входного файла.
Вы можете предполагать что входной файл содержит корректный вывод требуемой формулы.

\end{enumerate}

\end{document}
